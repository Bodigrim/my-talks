\documentclass[handout]{beamer}
\usepackage[utf8]{inputenc}
\usepackage[T2A]{fontenc}
\usepackage[english]{babel}
\usepackage{graphicx}
\usepackage{array}
\usetheme{Warsaw}
\usecolortheme{wolverine}

\usepackage{fontawesome5}
\usepackage{listings}
\usepackage[all]{xy}
\usepackage{changepage}

\definecolor{dkgreen}{rgb}{0,0.6,0}
\definecolor{gray}{rgb}{0.5,0.5,0.5}
\definecolor{mauve}{rgb}{0.58,0,0.82}

\lstdefinelanguage{Haskell}%
  {otherkeywords={},%
   morekeywords={as,abstype,if,then,else,case,class,data,default,deriving,%
      hiding,if,in,infix,infixl,infixr,import,instance,let,module,%
      newtype,of,qualified,type,where,do,AbsoluteSeek,AppendMode,%
      Array,BlockBuffering,BufferMode,Char,Complex,Double,%
      FilePath,Float,Word,Natural,IO,IOError,Ix,LineBuffering,Maybe,%
      Ordering,NoBuffering,ReadMode,ReadWriteMode,ReadS,RelativeSeek,%
      SeekFromEnd,SeekMode,ShowS,StdGen,String,Void,Bounded,Eq,%
      Eval,ExitCode,exitFailure,exitSuccess,Floating,Fractional,%
      Functor,Handle,HandlePosn,IOMode,Integral,List,Monad,MonadPlus,%
      MonadZero,Num,Numeric,Ord,Random,RandomGen,Ratio,Rational,Read,%
      Real,RealFloat,RealFrac,System,Prelude,EQ,GT,Just,%
      LT,Nothing,WriteMode,abs,accum,accumArray,%
      accumulate,acos,acosh,all,and,any,ap,appendFile,applyM,%
      approxRational,asTypeOf,asin,asinh,assocs,atan,atan2,atanh,%
      bounds,bracket,bracket_,break,catch,catMaybes,ceiling,chr,cis,%
      compare,concat,concatMap,conjugate,const,cos,cosh,curry,cycle,%
      decodeFloat,delete,deleteBy,deleteFirstsBy,denominator,%
      digitToInt,divMod,drop,dropWhile,either,elem,elems,elemIndex,%
      elemIndices,encodeFloat,error,exitFailure,exitWith,fail,%
      filter,filterM,find,findIndex,findIndices,flip,floatDigits,%
      floatRadix,floatRange,floatToDigits,floor,foldl,foldM,foldl1,%
      foldr,foldr1,fromDouble,fromInt,fromInteger,%
      toInteger,fromJust,fromMaybe,fromRat,fromRational,%
      fromRealFrac,fst,gcd,genericLength,genericTake,genericDrop,%
      genericSplitAt,genericIndex,genericReplicate,getArgs,getChar,%
      getContents,getEnv,getLine,getProgName,getStdGen,getStdRandom,%
      group,groupBy,guard,hClose,hFileSize,hFlush,hGetBuffering,%
      hGetChar,hGetContents,hGetLine,hGetPosn,hIsClosed,hIsEOF,hIsOpen,%
      hIsReadable,hIsSeekable,hIsWritable,hLookAhead,hPutChar,hPutStr,%
      hPutStrLn,hPrint,hReady,hSeek,hSetBuffering,hSetPosn,head,%
      hugsIsEOF,hugsHIsEOF,hugsIsSearchErr,hugsIsNameErr,%
      hugsIsWriteErr,id,ioError,imagPart,indices,init,inits,%
      inRange,insert,insertBy,interact,intersect,intersectBy,%
      intersperse,intToDigit,ioeGetErrorString,ioeGetFileName,%
      ioeGetHandle,isAlreadyExistsError,isAlreadyInUseError,isAlpha,%
      isAlphaNum,isAscii,isControl,isDenormalized,isDoesNotExistError,%
      isDigit,isEOF,isEOFError,isFullError,isHexDigit,isIEEE,%
      isIllegalOperation,isInfinite,isJust,isLower,isNaN,%
      isNegativeZero,isNothing,isOctDigit,isPermissionError,isPrefixOf,%
      isPrint,isSpace,isSuffixOf,isUpper,isUserError,ixmap,%
      join,last,lcm,lex,lexDigits,lexLitChar,liftM,liftM2,%
      liftM3,liftM4,liftM5,lines,listArray,listToMaybe,log,logBase,%
      lookup,magnitude,makePolar,mapAccumL,mapAccumR,mapAndUnzipM,%
      mapM,mapM_,mapMaybe,max,maxBound,maximum,maximumBy,maybe,%
      maybeToList,min,minBound,minimum,minimumBy,mkPolar,mkStdGen,%
      mplus,mod,msum,mzero,negate,next,newStdGen,notElem,nub,nubBy,%
      null,numerator,odd,openFile,or,ord,otherwise,partition,phase,pi,%
      polar,print,product,properFraction,putChar,putStr,putStrLn,%
      quot,random,randomIO,randomR,randomRIO,randomRs,randoms,%
      rangeSize,readDec,readFile,readFloat,readHex,readInt,readIO,%
      readList,readLitChar,readLn,readParen,readOct,readSigned,reads,%
      readsPrec,realPart,realToFrac,recip,rem,repeat,replicate,return,%
      reverse,round,scaleFloat,scanl,scanl1,scanr,scanr1,seq,sequence,%
      sequence_,setStdGen,show,showChar,showEFloat,showFFloat,%
      showFloat,showGFloat,showInt,showList,showLitChar,showParen,%
      showSigned,showString,shows,showsPrec,significand,signum,sin,%
      sinh,snd,sort,sortBy,span,split,splitAt,sqrt,stderr,stdin,stdout,%
      strict,subtract,sum,system,tail,tails,take,tan,%
      tanh,toInt,toInteger,toLower,toRational,toUpper,transpose,%
      truncate,try,uncurry,undefined,unfoldr,union,unionBy,unless,%
      unlines,until,unwords,unzip,unzip3,unzip4,unzip5,unzip6,unzip7,%
      userError,when,words,writeFile,zero,zip,zip3,zip4,zip5,zip6,zip7,%
      zipWith,zipWithM,zipWithM_,zipWith3,zipWith4,zipWith5,zipWith6,%
      zipWith7},%
   sensitive,%
   morecomment=[l]--,%
   morecomment=[n]{\{-}{-\}},%
   morestring=[b]"%
  }[keywords,comments,strings]%

\lstset{
  language=Haskell,
  showstringspaces=false,
  columns=flexible,
  keepspaces=true,
  basicstyle={\ttfamily},
  numbers=none,
  numberstyle=\tiny\color{gray},
  keywordstyle=\color{blue},
  commentstyle=\color{dkgreen},
  stringstyle=\color{mauve},
  escapeinside={<@}{@>},
  literate={μ}{{$\mu$}}1 {α}{{$\alpha$}}1 {->}{{$\to$}}2 {=>}{{$\Rightarrow$}}2 {<-}{{$\leftarrow$}}2 {≤}{{$\leqslant$}}1 {≥}{{$\geqslant$}}1 {∷}{{$::$}}1
}

\def\dd{\,.\,.\,}

\definecolor{darkblue}{rgb}{0,0,0.8}
\definecolor{darkgreen}{rgb}{0,0.6,0}
\definecolor{darkred}{rgb}{1,0,0}

\def\pros{\textcolor{darkgreen}{\bf Pros:} }
\def\cons{\textcolor{darkred}{\bf Cons:} }

\title{Bit vectors without compromises}
\author[Andrew Lelechenko]{Andrew Lelechenko \\ \texttt{1@dxdy.ru}}
\institute[Barclays]{Barclays, London}
\date{Haskell Love, 31.07.2020}

\begin{document}

\begin{frame}
  \titlepage
\end{frame}

\title{Shortest talk ever}

\begin{frame}{Bit vectors}

Bit vector is an array of booleans.

It can be represented as unboxed {\tt Vector Bool}.

\pause

\bigskip\bigskip\bigskip
\bigskip\bigskip\bigskip

\centerline{\Huge\bf Thank you!}

% \bigskip\bigskip\bigskip

% \centerline{\Huge $\mathcal{THE~~END}$}

\end{frame}

\title{Bit vectors without compromises}

\begin{frame}{Bit vectors as {\tt Vector Bool} from {\tt vector}}

\pros

\begin{itemize}
\item Allocates a continuous memory segment.
\item Random access is $O(1)$.
\item Has a mutable counterpart, so updates are $O(1)$.
\item Slicing is $O(1)$.
\item Loop fusion framework and rich API.
\end{itemize}

\pause

\cons

\begin{itemize}
\item Stores only 1 value per byte. \par
      So requires $8\times$ more space than theoretically possible.
\item Processes arrays bit by bit. \par
      So {\tt map} and {\tt zip} are $64\times$ slower than possible.
\end{itemize}

\end{frame}


\begin{frame}{Bit vectors as {\tt Array Bool} from {\tt array}}

\pros

\begin{itemize}
\item Allocates a continuous memory segment.
\item Random access is $O(1)$.
\item Has a mutable counterpart, so updates are $O(1)$.
\item Stores 64 values per {\tt Word64}. \par ~
\end{itemize}

\pause

\cons

\begin{itemize}
\item Processes arrays bit by bit. \par
      So {\tt map} and {\tt zip} are 64x slower than possible.
\item Slicing is $O(n)$.
\item No loop fusion framework and very limited API.
\end{itemize}

\end{frame}

\begin{frame}{Bit vectors as {\tt IntSet} from {\tt containers}}

\pros

\begin{itemize}
\item The best representation for sparse bit vectors.
\item Could store more than 8 values per {\tt Word64}.
\item Set operations are capable to process 64 elements at once.
\item Rich API.
\end{itemize}

\pause

\cons

\begin{itemize}
\item Employs a lot of pointers, not quite cache-friendly.
\item Random access is $O(\log n)$.
\item No mutable counterpart, so updates are $O(\log n)$. \par ~
\end{itemize}

\end{frame}

\begin{frame}{Bit vectors as {\tt Integer} from {\tt bv}}

\pros

\begin{itemize}
\item Allocates a continuous memory segment.
\item Random access is $O(1)$.
\item Stores 64 values per {\tt Word64}.
\item Some operations are capable process 64 elements at once.
\end{itemize}

\pause

\cons

\begin{itemize}
\item No mutable counterpart, so updates are \textcolor{darkred}{$O(n)$.}
      \par ~ \par ~ \par ~
\end{itemize}

\end{frame}


\begin{frame}[fragile]{No compromises}

\begin{itemize}[<+->]
\item Full-fledged {\tt Vector} and {\tt MVector} instances \par
      with expected asymptotic complexity.
\item Handy {\tt Bits} instance with vectorised blockwise operations.
\item As compact as possible: store 64 bits per {\tt Word64}.
\item Allocate a continuous memory segment.
\end{itemize}

\pause

\begin{lstlisting}[language=Haskell]
newtype Bit = Bit { unBit ∷ Bool }
data BitVec = BitVec
  { offset ∷ Int, -- in bits
  , length ∷ Int, -- in bits
  , array  ∷ ByteArray
  }
\end{lstlisting}

\pause

\begin{lstlisting}[language=Haskell]
index ∷ BitVec -> Int -> Bit
index (BitVec offset _ array) i =
  Bit (testBit (indexByteArray array q) r)
  where (q, r) = (i + offset) `quotRem` 64
\end{lstlisting}

\end{frame}

\begin{frame}[fragile]{Writing a mutable bit vector}

\begin{lstlisting}[language=Haskell]
data MBitVec s = MBitVec
  { offset ∷ Int, -- in bits
  , length ∷ Int, -- in bits
  , array  ∷ MutableByteArray s
  }
\end{lstlisting}

\pause

\begin{lstlisting}[language=Haskell]
write ∷ MBitVec s -> Int -> Bit -> ST s ()
write (MBitVec offset _ array) i (Bit b) = do
  let (q, r) = (i + offset) `quotRem` 64
  old <- readByteArray array q
  let new = (if b then setBit else clearBit) old r
  writeByteArray array q new
\end{lstlisting}

\pause

\bigskip

\centerline{\bf What could go wrong in a concurrent environment?}

\end{frame}

\begin{frame}[fragile]{Thread-safe writes}

Imagine having an atomic compare-and-swap (CAS):

\begin{lstlisting}[language=Haskell]
casArray ∷ MutableArray s a -> Int -> a -> a -> ST s a
casArray array offset expected new = do
  actual <- readByteArray array offset
  when (actual == expected) $
    writeByteArray array offset new
  pure actual
\end{lstlisting}

\pause

\begin{lstlisting}[language=Haskell]
write (MBitVec offset _ array) i (Bit b) =
  readByteArray array q >>= go
  where
    (q, r) = (i + offset) `quotRem` 64
    go expected = do
      let new = (if b then setBit else clearBit) expected r
      actual <- casArray array q expected new
      when (actual /= expected) (go actual)
\end{lstlisting}

\end{frame}

\begin{frame}[fragile]{Better thread-safe writes}

{\tt GHC.Exts} provides functions equivalent to

\begin{lstlisting}[language=Haskell]
fetchAndIntArray, fetchOrIntArray, fetchXorIntArray
  ∷ MutableByteArray s -> Int -> Int -> ST s Int
fetchAndIntArray array offset mask = ...
\end{lstlisting}

\pause

\begin{lstlisting}[language=Haskell]
write ∷ MBitVec s -> Int -> Bit -> ST s ()
write (MBitVec offset _ array) i (Bit b) = if b
  then fetchOrIntArray  array q             (bit r)
  else fetchAndIntArray array q (complement (bit r))
  where (q, r) = (i + offset) `quotRem` 64
\end{lstlisting}

\end{frame}

\begin{frame}[fragile]{Modifying a mutable bit vector}

\begin{lstlisting}[language=Haskell]
modify ∷ MVector s a -> (a -> a) -> Int -> ST s ()
modify vec func offset = ...
\end{lstlisting}

\pause

\begin{lstlisting}[language=Haskell]
modify ∷ MVector s Bit -> (Bit -> Bit) -> Int -> ST s ()
\end{lstlisting}

\pause

\bigskip

There are only 4 functions {\tt Bit}${}\to{}${\tt Bit}:
\begin{itemize}[<+->]
\item {\tt id},
\item {\tt const True},
\item {\tt const False},
\item \textcolor{darkred}{\tt not}.
\end{itemize}

\end{frame}

\begin{frame}[fragile]{Flipping a bit}

\begin{lstlisting}[language=Haskell]
flipBit ∷ MVector s Bit -> Int -> ST s ()
flipBit vec i = do
  Bit b <- read vec i
  write vec i (not b)
\end{lstlisting}

\pause

\centerline{\bf What could go wrong in a concurrent environment?}

\pause

\bigskip
\bigskip

\begin{lstlisting}[language=Haskell]
flipBit ∷ MVector s Bit -> Int -> ST s ()
flipBit (MBitVec offset _ array) i =
  fetchXorIntArray array q (bit r)
  where (q, r) = (i + offset) `quotRem` 64
\end{lstlisting}

\pause

\medskip

Killing two birds with one stone:

\begin{itemize}[<+->]
\item Faster!
\item Thread-safe!
\end{itemize}

\end{frame}

\begin{frame}{Test your unboxed vectors}

\begin{itemize}[<+->]
\item {\tt MVector} interface requires also defining
      {\tt copy}, {\tt move}, {\tt set}, {\tt grow},
      all dealing correctly with (possibly, unaligned) offsets and lengths.
\item Covering all cases with unit tests is unfeasible.
\item {\tt bitvec-0.1} was notoriously buggy.
\item {\tt Test.QuickCheck.Classes.muvectorLaws} provides $\sim30$~properties
      for thorough testing of unboxed mutable vectors.
\item Also available from {\tt Hedgehog.Classes.muvectorLaws}.
\item These properties have proved to be useful in test suites
      of {\tt bitvec}, {\tt arithmoi}, {\tt mod}\dots
\end{itemize}

\end{frame}

\begin{frame}[fragile]{Blockwise {\tt map}}

\begin{lstlisting}[language=Haskell]
map ∷ (a   -> a)   -> Vector a   -> Vector a
map ∷ (Bit -> Bit) -> Vector Bit -> Vector Bit
\end{lstlisting}

\pause
\bigskip

There are only 4 functions {\tt Bit}${}\to{}${\tt Bit}:
\begin{itemize}
\item {\tt id},
\item {\tt const True},
\item {\tt const False},
\item \textcolor{darkred}{\tt not}.
\end{itemize}

\pause

\begin{lstlisting}[language=Haskell]
invertBits ∷ Vector Bit -> Vector Bit
\end{lstlisting}

\pause

\begin{itemize}[<+->]
\item Invert 64 bits at once, applying {\tt complement} to {\tt Word64}.
\item Take extra care for unaligned vectors in concurrent environment.
\item Use {\tt mpn\_com} from GMP for ultra-fast vectorised processing, \par
      up to $1000\times$ faster than {\tt Vector Bool}.
\end{itemize}

\end{frame}

\begin{frame}[fragile]{Blockwise {\tt zip}}

\begin{lstlisting}[language=Haskell]
zip ∷ (a -> a -> a) -> Vector a -> Vector a -> Vector a
zip ∷ (Bit -> Bit -> Bit)
    -> Vector Bit -> Vector Bit -> Vector Bit
\end{lstlisting}

\pause

\bigskip

There are only 16 functions {\tt Bit}${}\to{}${\tt Bit}${}\to{}${\tt Bit}:

\begin{itemize}[<+->]
\item {\tt True}, {\tt False}, $x$, $\bar x$, $y$, $\bar y$,
\item $ x \wedge y $, $\bar x \wedge y$, $x \wedge \bar y$, $\bar x \wedge \bar y$,
\item $ x \vee y $, $\bar x \vee y$, $x \vee \bar y$, $\bar x \vee \bar y$,
\item $ x + y $, $\bar x + y$, \textcolor{red}{$x + \bar y$, $\bar x + \bar y$.}
\end{itemize}

\bigskip

\begin{itemize}[<+->]
\item Zip 64 bits at once, applying {\tt complement},
      {\tt .\&.}, {\tt .|.} and {\tt xor} on {\tt Word64}.
\item Aligning two vectors with different offsets is tough by itself,
      and becomes even worse in concurrent environment.
\item Use rountines from GMP for ultra-fast vectorised processing, \par
      up to $1000\times$ faster than {\tt Vector Bool}.
\end{itemize}

\end{frame}

\begin{frame}{Additional goodness in {\tt bitvec}}

\begin{itemize}[<+->]
\item Blockwise population count and its reverse {\tt nthBitIndex}. \par ~
\item Operations for succinct data structures, backed by BMI2 instructions. \par ~
\item Ultra-fast reversal, up to $O(1)$. \par ~
\item Boolean polynomials for cryptographic applications. \par ~
\item Conversions from/to {\tt Vector Word} and {\tt ByteString}.
\end{itemize}

\end{frame}

\begin{frame}{Bit vectors without compromises}

\begin{itemize}[<+->]
\item Full-fledged {\tt Vector} and {\tt MVector} instances \par
      with expected asymptotic complexity \par
      (but constant factor is up to 20\% larger).
\item Handy {\tt Bits} instance with vectorised blockwise operations \par
      (usually $64\times$ and up to $1000\times$ faster).
\item Allocate $8\times$ less memory than {\tt Vector Bool}.
\end{itemize}

\bigskip
\bigskip
\bigskip

\pause

\centerline{\Huge\bf Thank you!}

\bigskip
\bigskip

\centerline{
\faAt\ 1@dxdy.ru ~~ \par \faTelegram\ Bodigrim ~~~\,
}

\medskip

\centerline{
\par \faGithub\ github.com/Bodigrim/bitvec ~~
\par \faGithub\ github.com/Bodigrim/my-talks
}

\end{frame}

\end{document}
