\documentclass[handout]{beamer}
\usepackage[utf8]{inputenc}
\usepackage[T2A]{fontenc}
\usepackage[english]{babel}
\usepackage{graphicx}
\usepackage{array}
\usetheme{Warsaw}
\usecolortheme{wolverine}

\usepackage{fontawesome5}
\usepackage{listings}
\usepackage[all]{xy}
\usepackage{changepage}

\definecolor{dkgreen}{rgb}{0,0.6,0}
\definecolor{gray}{rgb}{0.5,0.5,0.5}
\definecolor{mauve}{rgb}{0.58,0,0.82}

\lstdefinelanguage{Haskell}%
  {otherkeywords={},%
   morekeywords={as,abstype,if,then,else,case,class,data,default,deriving,%
      hiding,if,in,infix,infixl,infixr,import,instance,let,module,%
      newtype,of,qualified,type,where,do,AbsoluteSeek,AppendMode,%
      Array,BlockBuffering,BufferMode,Char,Complex,Double,%
      FilePath,Float,Word,Natural,IO,IOError,Ix,LineBuffering,Maybe,%
      Ordering,NoBuffering,ReadMode,ReadWriteMode,ReadS,RelativeSeek,%
      SeekFromEnd,SeekMode,ShowS,StdGen,String,Void,Bounded,Eq,%
      Eval,ExitCode,exitFailure,exitSuccess,Floating,Fractional,%
      Functor,Handle,HandlePosn,IOMode,Integral,List,Monad,MonadPlus,%
      MonadZero,Num,Numeric,Ord,Random,RandomGen,Ratio,Rational,Read,%
      Real,RealFloat,RealFrac,System,Prelude,EQ,GT,Just,%
      LT,Nothing,WriteMode,abs,accum,accumArray,%
      accumulate,acos,acosh,all,and,any,ap,appendFile,applyM,%
      approxRational,array,asTypeOf,asin,asinh,assocs,atan,atan2,atanh,%
      bounds,bracket,bracket_,break,catch,catMaybes,ceiling,chr,cis,%
      compare,concat,concatMap,conjugate,const,cos,cosh,curry,cycle,%
      decodeFloat,delete,deleteBy,deleteFirstsBy,denominator,%
      digitToInt,divMod,drop,dropWhile,either,elem,elems,elemIndex,%
      elemIndices,encodeFloat,error,exitFailure,exitWith,fail,%
      filter,filterM,find,findIndex,findIndices,flip,floatDigits,%
      floatRadix,floatRange,floatToDigits,floor,foldl,foldM,foldl1,%
      foldr,foldr1,fromDouble,fromInt,fromInteger,%
      toInteger,fromJust,fromMaybe,fromRat,fromRational,%
      fromRealFrac,fst,gcd,genericLength,genericTake,genericDrop,%
      genericSplitAt,genericIndex,genericReplicate,getArgs,getChar,%
      getContents,getEnv,getLine,getProgName,getStdGen,getStdRandom,%
      group,groupBy,guard,hClose,hFileSize,hFlush,hGetBuffering,%
      hGetChar,hGetContents,hGetLine,hGetPosn,hIsClosed,hIsEOF,hIsOpen,%
      hIsReadable,hIsSeekable,hIsWritable,hLookAhead,hPutChar,hPutStr,%
      hPutStrLn,hPrint,hReady,hSeek,hSetBuffering,hSetPosn,head,%
      hugsIsEOF,hugsHIsEOF,hugsIsSearchErr,hugsIsNameErr,%
      hugsIsWriteErr,id,ioError,imagPart,indices,init,inits,%
      inRange,insert,insertBy,interact,intersect,intersectBy,%
      intersperse,intToDigit,ioeGetErrorString,ioeGetFileName,%
      ioeGetHandle,isAlreadyExistsError,isAlreadyInUseError,isAlpha,%
      isAlphaNum,isAscii,isControl,isDenormalized,isDoesNotExistError,%
      isDigit,isEOF,isEOFError,isFullError,isHexDigit,isIEEE,%
      isIllegalOperation,isInfinite,isJust,isLower,isNaN,%
      isNegativeZero,isNothing,isOctDigit,isPermissionError,isPrefixOf,%
      isPrint,isSpace,isSuffixOf,isUpper,isUserError,ixmap,%
      join,last,lcm,length,lex,lexDigits,lexLitChar,liftM,liftM2,%
      liftM3,liftM4,liftM5,lines,listArray,listToMaybe,log,logBase,%
      lookup,magnitude,makePolar,mapAccumL,mapAccumR,mapAndUnzipM,%
      mapM,mapM_,mapMaybe,max,maxBound,maximum,maximumBy,maybe,%
      maybeToList,min,minBound,minimum,minimumBy,mkPolar,mkStdGen,%
      mplus,mod,msum,mzero,negate,next,newStdGen,not,notElem,nub,nubBy,%
      null,numerator,odd,openFile,or,ord,otherwise,partition,phase,pi,%
      polar,print,product,properFraction,putChar,putStr,putStrLn,%
      quot,random,randomIO,randomR,randomRIO,randomRs,randoms,%
      rangeSize,read,readDec,readFile,readFloat,readHex,readInt,readIO,%
      readList,readLitChar,readLn,readParen,readOct,readSigned,reads,%
      readsPrec,realPart,realToFrac,recip,rem,repeat,replicate,return,%
      reverse,round,scaleFloat,scanl,scanl1,scanr,scanr1,seq,sequence,%
      sequence_,setStdGen,show,showChar,showEFloat,showFFloat,%
      showFloat,showGFloat,showInt,showList,showLitChar,showParen,%
      showSigned,showString,shows,showsPrec,significand,signum,sin,%
      sinh,snd,sort,sortBy,span,split,splitAt,sqrt,stderr,stdin,stdout,%
      strict,subtract,sum,system,tail,tails,take,tan,%
      tanh,toInt,toInteger,toLower,toRational,toUpper,transpose,%
      truncate,try,uncurry,undefined,unfoldr,union,unionBy,unless,%
      unlines,until,unwords,unzip,unzip3,unzip4,unzip5,unzip6,unzip7,%
      userError,when,words,writeFile,zero,zip,zip3,zip4,zip5,zip6,zip7,%
      zipWith,zipWithM,zipWithM_,zipWith3,zipWith4,zipWith5,zipWith6,%
      zipWith7},%
   sensitive,%
   morecomment=[l]--,%
   morecomment=[n]{\{-}{-\}},%
   morestring=[b]"%
  }[keywords,comments,strings]%

\lstset{
  language=Haskell,
  showstringspaces=false,
  columns=flexible,
  keepspaces=true,
  basicstyle={\ttfamily},
  numbers=none,
  numberstyle=\tiny\color{gray},
  keywordstyle=\color{blue},
  commentstyle=\color{dkgreen},
  stringstyle=\color{mauve},
  escapeinside={<@}{@>},
  literate={μ}{{$\mu$}}1 {α}{{$\alpha$}}1 {->}{{$\to$}}2 {=>}{{$\Rightarrow$}}2 {<-}{{$\leftarrow$}}2 {≤}{{$\leqslant$}}1 {≥}{{$\geqslant$}}1
}

\def\dd{\,.\,.\,}

\title{Enum instances on steroids}
\author[Andrew Lelechenko]{Andrew Lelechenko \\ \texttt{1@dxdy.ru}}
\institute[Barclays]{Barclays, London}
\date{Berlin Functional Programming Group, 21.07.2020}

\begin{document}

\begin{frame}
  \titlepage
\end{frame}

\begin{frame}{List generators in Haskell}

\begin{itemize}[<+->]

\item Generate a finite range:
$$
\phantom{,3} [1\dd5] = [1,2,3,4,5].\phantom{\ldots}
$$

\item Generate an infinite range:
$$
\phantom{,3} [1\dd] \phantom{5} = [1,2,3,4,5\ldots].
$$

\item Generate ranges with a step:
\begin{align*}
[1,3\dd9] &= [1,3,5,7,9], \\
[1,3\dd] \phantom{9} &= [1,3,5,7,9\ldots]. \\
\end{align*}
\end{itemize}

\end{frame}

\begin{frame}{Double trouble}

\begin{itemize}[<+->]

\item As expected,
$$[0,1\dd3] \phantom{.5} = [0,1,2,3]. \phantom{,4} $$

\item But
$$ [0,1\dd3.5] = [0,1,2,3,4]. $$

\centerline{\bf Why?}

\bigskip

\item {\em Short answer:} because Haskell Report 2010 says so!

\item {\em Longer answer:} floating-point arithmetic is not exact, so
      without some kind of rounding we'd have equally surprising
      $$ [0,0.1\dd0.3] = [0,0.1,0.2], $$
      because
      $$ 0.1+0.1+0.1 = 0.30000000000000004 > 0.3. $$

\end{itemize}

\end{frame}

\begin{frame}{Not only numbers}

\begin{itemize}[<+->]

\item List generators are available for user-defined types:
$$
data~ \text{Day} =
\text{Mon} \mid \text{Tue} \mid \text{Wed} \mid \text{Thu} \mid \text{Fri} \mid \text{Sat} \mid \text{Sun}
{~deriving~} \text{Enum}
$$

\item Generate a range:
$$
[\text{Tue}\dd \text{Fri}] = [\text{Tue}, \text{Wed}, \text{Thu}, \text{Fri}].
$$

\item Generate a range with a step:
$$
[\text{Mon},\text{Wed}\dd\text{Sun}] = [\text{Mon},\text{Wed},\text{Fri},\text{Sun}].
$$

\end{itemize}

\end{frame}

\begin{frame}[fragile]{What happens under the hood?}

\begin{lstlisting}[language=Haskell]
class Enum a where
  enumFrom       :: a -> [a]
                 -- 1 -> [1,2,3,4,5...]
  enumFromThen   :: a -> a -> [a]
                 -- 1 -> 3 -> [1,3,5,7,9...]
  enumFromTo     :: a -> a -> [a]
                 -- 1 -> 5 -> [1,2,3,4,5]
  enumFromThenTo :: a -> a -> a -> [a]
                 -- 1 -> 3 -> 9 -> [1,3,5,7,9]
\end{lstlisting}

\end{frame}

\begin{frame}[fragile]{We need to go deeper}

\begin{lstlisting}[language=Haskell]
class Enum a where
  fromEnum :: a -> Int
  toEnum   :: Int -> a
\end{lstlisting} \pause
\begin{lstlisting}[language=Haskell]
  succ :: a -> a -- next
  succ x = toEnum (fromEnum x + 1)

  pred :: a -> a -- prev
  pred x = toEnum (fromEnum x - 1)
\end{lstlisting} \pause
\begin{lstlisting}[language=Haskell]
  enumFrom :: a -> [a]
  enumFrom x = map toEnum (iterate (+ 1) (fromEnum x))

  enumFromTo :: a -> a -> [a]
  enumFromTo x y = map toEnum
    (takeWhile (≤ fromEnum y) (iterate (+ 1) (fromEnum x)))
\end{lstlisting}

\end{frame}

\begin{frame}{Partial functions}

\begin{itemize}[<+->]

\item {\tt succ maxBound} / {\tt pred minBound} are undefined.

\bigskip

\item Most of enumerable data is ``smaller'' than {\tt Int},
      \par so $ \texttt{toEnum} :: \texttt{Int} \to a $ is a partial function.
\item Others ({\tt Integer}, {\tt Natural}) are ``bigger'',
      \par so $ \texttt{fromEnum} :: a \to \texttt{Int} $ has a truncating behavior.
\item The only faithful instance is {\tt Enum Int}.

\bigskip

\item
{\tt fromEnum} and {\tt toEnum} have utility, reaching far beyond
list comprehensions:

\begin{itemize}[<+->]
  \item Generating random values.
  \item Faster maps and sets, backed by {\tt IntMap} and {\tt IntSet}.
  \item Storing values in unboxed vectors.
\end{itemize}

\end{itemize}
\end{frame}

\begin{frame}{No built-in deriving}

{\bf Biggest issue:} {\tt Enum} is derivable only for types,
      whose constructors have no fields. This does not work:

\begin{align*}
data~ \text{WeekDay} &= \text{Mon} \mid \text{Tue} \mid \text{Wed} \mid \text{Thu} \mid \text{Fri} &{~deriving~} \text{Enum} \\
data~ \text{WeekEnd} &= \text{Sat} \mid \text{Sun} &{~deriving~} \text{Enum} \\
data~ \text{Day} &= \text{Work~WeekDay} \mid \text{Play WeekEnd} &{~deriving~} \text{\textcolor{red}{Enum}}
\end{align*}

\pause

This also does not work:
\begin{align*}
[ {\rm Nothing} \dd {\rm Just~True} ] &:: [{\rm Maybe~Bool}] \\
[ {\rm Left~}() \dd {\rm Right~True} ] &:: [{\rm Either~()~Bool}]
\end{align*}

\end{frame}

\begin{frame}[fragile]{Cardinality}

\begin{itemize}[<+->]

\item
{\em Cardinality} is just a fancy word for the number of values
inhabiting a type.

\begin{itemize}[<+->]
  \item Cardinality of {\tt Bool} is 2.
  \item Cardinality of $()$~~~~ is 1.
  \item Cardinality of {\tt Void} is 0.
\end{itemize}

\item Cardinality of
{\tt data Foo = Foo Bool Bool | Bar () }
equals to $2\times2 + 1 = 5$.

\item
In its essence {\tt Enum} is about finitely-inhabited types,
however it lacks means to infer their cardinality,
which makes it unreliable and its instances difficult to define.

\pause

\begin{lstlisting}[language=Haskell]
class MyEnum a where
  cardinality :: Proxy a -> Integer
  toMyEnum    :: Integer -> a
  fromMyEnum  :: a -> Integer
\end{lstlisting}

\end{itemize}

\end{frame}

\begin{frame}{Products and sums of types}

\begin{itemize}[<+->]

\item Each {\em algebraic} data type is isomorphic to a sum of products of
its constituents, which is called its {\em generic} representation.

\item
{\tt data Foo = Foo Bool Bool} is isomorphic to {\tt (Bool, Bool)}
or in other notation $ \text{Bool} {{}:}{\times}{:{}} \text{Bool} $.

\item
{\tt data Foo = Bar Bar | Baz Baz} is isomorphic to {\tt Either Bar Baz}
or in other notation $ \text{Bar} {{}:}{+}{:{}} \text{Baz} $.

\item
{\tt data Foo = Foo} is isomorphic to $()$ type.

\item
{\tt data Foo = Foo Bool Bool | Bar} is isomorphic to
{\tt Either (Bool, Bool) ()} or in other notation $ \text{Bool} {{}:}{\times}{:{}} \text{Bool} {{}:}{+}{:{}} () $.

\item
GHC provides an automatic way to convert between types
and their generic representations.

\end{itemize}

\end{frame}

\begin{frame}[fragile]{Generic instance for sum}

\begin{lstlisting}[language=Haskell]
class GMyEnum f where
  gcardinality :: Proxy f -> Integer
  toGMyEnum    :: Integer -> f a
  fromGMyEnum  :: f a -> Integer
\end{lstlisting} \pause
\begin{lstlisting}[language=Haskell]
instance (GMyEnum a, GMyEnum b) => GMyEnum (a :+: b) where
  gcardinality _ =
    gcardinality (Proxy @a) + gcardinality (Proxy @b)
\end{lstlisting} \pause
\begin{lstlisting}[language=Haskell]
  toGMyEnum n | n < cardA = L1 (toGMyEnum n)
              | otherwise = R1 (toGMyEnum (n - cardA))
    where cardA = gcardinality (Proxy @a)
\end{lstlisting} \pause
\begin{lstlisting}[language=Haskell]
  fromGMyEnum = \case
     L1 x -> fromGMyEnum x
     R1 x -> fromGMyEnum x + gcardinality (Proxy @a)
\end{lstlisting}

\end{frame}

\begin{frame}[fragile]{Generic instance for product}

\begin{lstlisting}[language=Haskell]
instance (GMyEnum a, GMyEnum b) => GMyEnum (a :*: b) where
  gcardinality _ =
    gcardinality (Proxy @a) * gcardinality (Proxy @b)
\end{lstlisting} \pause
\begin{lstlisting}[language=Haskell]
  toGMyEnum n = toGMyEnum q :*: toGMyEnum r
    where
      cardB = gcardinality (Proxy @b)
      (q, r) = n `quotRem` cardB
\end{lstlisting} \pause
\begin{lstlisting}[language=Haskell]
  fromGMyEnum (q :*: r) =
    gcardinality (Proxy @b) * fromGMyEnum q + fromGMyEnum r

\end{lstlisting}

\bigskip\pause

Full source code available from
\url{https://github.com/Bodigrim/random/blob/generic/src/System/Random/GFinite.hs}

\end{frame}

\begin{frame}[fragile]{Example of autoderiving}

\begin{lstlisting}[language=Haskell]
{-# LANGUAGE DeriveGeneric  #-}
{-# LANGUAGE DeriveAnyClass #-}

import GHC.Generics

data Action = Code Bool | Eat Bool Bool | Sleep ()
  deriving (Show, Generic, MyEnum)
\end{lstlisting} \pause
\begin{lstlisting}[language=Haskell]
> cardinality (Proxy @Action)
7
\end{lstlisting} \pause
\begin{lstlisting}[language=Haskell]
> map toMyEnum [0..7-1] :: [Action]
[Code False,Code True,Eat False False,Eat False True,
 Eat True False,Eat True True,Sleep ()]
\end{lstlisting}

\end{frame}

\begin{frame}[fragile]{Making illegal states unrepresentable}

\begin{lstlisting}[language=Haskell]
class MyEnum a where
  cardinality :: Proxy a -> Integer
  toMyEnum    :: Integer -> a
  fromMyEnum  :: a -> Integer
\end{lstlisting}

\pause

Define {\tt data Finite} $(n :: \texttt{Nat})$, which is inhabited exactly by~$n$~values $[0\dd n-1]$,
and promote cardinality to the type level:

\begin{lstlisting}[language=Haskell]
class MyEnum a where
  type Cardinality a :: Nat
  toMyEnum   :: Finite (Cardinality a) -> a
  fromMyEnum :: a -> Finite (Cardinality a)
\end{lstlisting}

\pause

This approach can be found in \href{http://hackage.haskell.org/package/finitary}{\tt finitary} and \href{http://hackage.haskell.org/package/finitary-derive}{\tt finitary-derive}.

\end{frame}

\begin{frame}[fragile]{Countable\dots}

\begin{itemize}[<+->]

\item
A data type is called countable if it is isomorphic to {\tt Integer}: there exist total
functions

\begin{lstlisting}[language=Haskell]
fromCountable :: a -> Integer
toCountable   :: Integer -> a
\end{lstlisting}

\item
{\tt Either Integer Integer} is still countable: map {\tt Left} to even numbers
and {\tt Right} to odd numbers.

\begin{lstlisting}[language=Haskell]
fromCountable :: Either Integer Integer -> Integer
fromCountable (Left n)  = n * 2
fromCountable (Right n) = n * 2 + 1

toCountable :: Integer -> Either Integer Integer
toCountable n
  | even n    = Left  ( n      `div` 2)
  | otherwise = Right ((n - 1) `div` 2)
\end{lstlisting}

\end{itemize}

\end{frame}

\begin{frame}[fragile]{\dots and uncountable}

\begin{itemize}[<+->]

\item
{\tt (Integer, Integer)} is also countable: just interleave bits from both coordinates.

\begin{lstlisting}[language=Haskell]
fromCountable :: (Integer, Integer) -> Integer
fromCountable (0b1111, 0b0000) = 0b10101010

toCountable :: Integer -> (Integer, Integer)
toCountable 0b10101010 = (0b1111, 0b0000)
\end{lstlisting}

\item So sums and products of countable data are countable again!
      {\bf But what is uncountable then?}

\item Set of functions $\texttt{Integer} \to \texttt{Bool}$ is uncountable,
      isomorphic to the set of real numbers.

\end{itemize}

\end{frame}

\begin{frame}[fragile]{Extending {\tt MyEnum} for countable data}

\begin{itemize}[<+->]

\item Define

\begin{lstlisting}[language=Haskell]
data Cardinality = Finite Integer | Countable

class MyEnum a where
  cardinality :: Proxy a -> Cardinality
  toMyEnum    :: Cardinality -> a
  fromMyEnum  :: a -> Cardinality
\end{lstlisting}

\item This makes infinitely-inhabited types such as {\tt [Bool]} ``enumerable''.

\item
Implemented in
\href{http://hackage.haskell.org/package/cantor-pairing}{\tt cantor-pairing} package.

\end{itemize}

\end{frame}

\begin{frame}

\centerline{\Huge\bf Thank you!}

\bigskip
\bigskip

\centerline{
\par \faAt\ andrew.lelechenko@gmail.com ~~ \faTelegram\ Bodigrim
}

\bigskip

\centerline{
\par \faGithub\ github.com/Bodigrim/my-talks
}

\end{frame}

\end{document}
