\documentclass[handout]{beamer}
\usepackage[utf8]{inputenc}
\usepackage[T2A]{fontenc}
\usepackage[english]{babel}
\usepackage{graphicx}
\usepackage{array}
\usetheme{Warsaw}
\usecolortheme{wolverine}

\usepackage{fontawesome5}
\usepackage{listings}
\usepackage[all]{xy}
\usepackage{changepage}
\usepackage{dirtree}

\definecolor{dkgreen}{rgb}{0,0.6,0}
\definecolor{gray}{rgb}{0.5,0.5,0.5}
\definecolor{mauve}{rgb}{0.58,0,0.82}

\lstdefinelanguage{Haskell}%
  {otherkeywords={},%
   morekeywords={as,case,class,data,default,deriving,do,else,family,forall,foreign,hiding,if,import,in,infix,infixl,infixr,instance,let,mdo,module,newtype,of,proc,qualified,rec,then,type,where},%
   sensitive,%
   morecomment=[l]--,%
   morecomment=[n]{\{-}{-\}},%
   morestring=[b]"%
  }[keywords,comments,strings]%

\lstset{
  language=Haskell,
  showstringspaces=false,
  columns=flexible,
  keepspaces=true,
  basicstyle={\ttfamily},
  numbers=none,
  numberstyle=\tiny\color{gray},
  keywordstyle=\color{blue},
  commentstyle=\color{dkgreen},
  stringstyle=\color{mauve},
  escapeinside={<@}{@>},
  literate={μ}{{$\mu$}}1 {α}{{$\alpha$}}1 {->}{{$\to$}}2 {=>}{{$\Rightarrow$}}2 {<-}{{$\leftarrow$}}2 {≤}{{$\leqslant$}}1 {<=}{{$\leqslant$}}1 {≥}{{$\geqslant$}}1 {>=}{{$\geqslant$}}1 {∷}{{$::$}}1 {::}{{$::$}}1 {...}{{\dots}}1
}

\def\dd{\,.\,.\,}
\def\ge{\geqslant}
\def\le{\leqslant}

\DeclareMathOperator{\quotRem}{quotRem}

\definecolor{darkblue}{rgb}{0,0,0.8}
\definecolor{darkgreen}{rgb}{0,0.6,0}
\definecolor{darkred}{rgb}{1,0,0}

\def\pros{\textcolor{darkgreen}{\bf Pros:} }
\def\cons{\textcolor{darkred}{\bf Cons:} }

\title{Polynomials in Haskell}
\author[Andrew Lelechenko]{Andrew Lelechenko \\ \texttt{1@dxdy.ru}}
\institute[Barclays]{Barclays, London}
\date{MuniHac, 12.09.2020}

\begin{document}

\begin{frame}
  \titlepage
\end{frame}

\begin{frame}{Univariate polynomials}
\begin{definition}
A polynomial is
\begin{itemize}
\item either a monomial $c\cdot x^k$ for $k \ge 0$,
\item or a sum of two polynomials.
\end{itemize}
\end{definition}

\pause

\begin{itemize}[<+->]
\item $2x^3$, $x^2$, $4$ are polynomials.
\item $1/x$ is not a polynomial.
\item $4+2x^3$ and $x+x^2$ are polynomials as well.
\item Polynomials can be added and multiplied.

\vspace{-1em}

\begin{align*}
(x + 1) + (x - 1) &= 2 x, \\
(x + 1) \times (x - 1) &= x^2 - 1.
\end{align*}

\end{itemize}

\end{frame}

\begin{frame}{Why Haskell?}
\begin{itemize}[<+->]
\item
\textcolor{darkred}{\bf Question:} \par
There are efficient Fortran libraries
for polynomial manipulations. Can we use them?
\item
\textcolor{darkgreen}{\bf Answer:} \par
Not really, Fortran libraries are mostly
tied to {\tt Double}.
\item My main applications are algebra and
cryptography, where coefficients are discrete
(think of {\tt Integer} and modular arithmetic).
\item We would like our implementation of polynomials
to be polymorphic in coefficients.
\item Haskell has a good track record for polymorphism,
but a modest one for performance.
\end{itemize}
\end{frame}

\begin{frame}[fragile]{Haskell translation}

\begin{lstlisting}[language=Haskell]
data Poly a
  = X { power :: Word, coeff :: a }
  | Poly a :+ Poly a
  deriving (Eq)
\end{lstlisting}

\pause

\begin{lstlisting}[language=Haskell]
instance Num a => Num (Poly a) where
  fromInteger n = X 0 (fromInteger n)
  (+) = (:+)

  (x :+ y) * z = (x * z) :+ (y * z)
  x * (y :+ z) = (x * y) :+ (x * z)
  X k c * X l d = X (k + l) (c * d)
\end{lstlisting}

\medskip

\centerline{\bf Beautiful, innit?}

\bigskip

\pause

However polynomial addition, as defined, is not associative:
$$ x + (y + z) \ne (x + y) + z. $$

\end{frame}

\begin{frame}[fragile]{Fixing associativity}

\begin{itemize}[<+->]
\item There are some basic blocks (monomials).
\item There is an associative operation (addition) \par
      with a neutral element (monomial with zero coefficient).
\item Thus, polynomials are monoids over monomials.
\item Lists are {\bf free} monoids,
      let's use them to  model a non-free one.
\end{itemize}

\pause

\begin{lstlisting}[language=Haskell]
data Mono a = X { power :: Word, coeff :: a }
  deriving (Eq, Ord)
mul :: Num a => Mono a -> Mono a -> Mono a
mul (X k c) (X l d) = X (k + l) (c * d)
\end{lstlisting}

\pause

\begin{lstlisting}[language=Haskell]
newtype Poly a = Poly [Mono a] deriving (Eq)
instance Num a => Num (Poly a) where
  Poly xs + Poly ys = Poly (xs <> ys)
  Poly xs * Poly ys = foldl (+) (Poly [])
    (map (\x -> Poly (map (mul x) ys)) xs)
\end{lstlisting}

\medskip

\centerline{\bf Better?}
\end{frame}

\begin{frame}[fragile]{Fixing commutativity}

Not quite: polynomial addition, as defined, is not commutative:
$$ x + y \ne y + x. $$

\pause

Commutativity means that all rearrangements of {\tt [Mono a]} must be equivalent.

\begin{lstlisting}[language=Haskell]
data Mono a = X { power :: Word, coeff :: a }
  deriving (Eq, Ord)

newtype Poly a = Poly [Mono a]
instance Ord a => Eq (Poly a) where
  (==) = (==) `on` sort
\end{lstlisting}

\pause

{\bf No!}
Structural equality is too important to be sacrificed.
\end{frame}

\begin{frame}[fragile]{Sorted lists}

\begin{lstlisting}[language=Haskell]
newtype Sorted a = Sorted { unSorted :: [a] }

sorted :: [a] -> Sorted a
sorted = Sorted . sort

merge :: Ord a => [a] -> [a] -> [a]
merge xs [] = xs
merge [] ys = ys
merge (x : xs) (y : ys)
  | x <= y    = x : merge xs (y : ys)
  | otherwise = y : merge (x : xs) ys
\end{lstlisting}


\medskip

\centerline{\bf All sorted?}

\bigskip

\pause

Not yet: we would also expect that $x+x = 2x$ and $0x = 0$.

\end{frame}

\begin{frame}[fragile]{Final touches}

\begin{lstlisting}[language=Haskell]
merge :: (Eq a, Num a) => [Mono a] -> [Mono a] -> [Mono a]
merge xs [] = xs
merge [] ys = ys
merge (X k c : xs) (X l d : ys) = case k `compare` l of
  LT -> X k c : merge xs (X l d : ys)
  EQ -> case c + d of
    0 -> merge xs ys
    e -> X k e : merge xs ys
  GT -> Y l d : merge (X k c : xs) (ys)
\end{lstlisting}

\pause

\begin{lstlisting}[language=Haskell]
newtype Poly a = Poly [Mono a] deriving (Eq)
instance (Eq a, Num a) => Num (Poly a) where
  Poly xs + Poly ys = Poly (xs `merge` ys)
  Poly xs * Poly ys = foldl (+) (Poly [])
    (map (\x -> Poly (map (mul x) ys)) xs)
\end{lstlisting}

\medskip

\centerline{\bf Structural equality is restored!}

\end{frame}

\begin{frame}[fragile]{Memory representation}

\begin{itemize}[<+->]
\item {\tt Mono a} takes 5+ words: tag, pointer to a power and
      its value, pointer to a coefficient and its value.
\item {\tt [Mono a]} takes 8+ words per monomial.
\item If {\tt a} is {\tt Double} or {\tt Int},
      this is very expensive.
\item Lazy lists accumulate long sequence of thunks,
      especially in multiplication.
\item \textcolor{darkgreen}{\bf Solution:}
      use vectors!
\item Shall we use boxed or unboxed vectors?
\item \textcolor{darkgreen}{\bf Solution:}
      be polymorphic by vector flavour.
\end{itemize}

\pause

\begin{lstlisting}[language=Haskell]
newtype Poly v a = Poly (v (Word, a)) deriving (Eq)
instance (Eq a, Num a, Vector v (Word, a)) => Num (Poly v a)
\end{lstlisting}

\medskip

\centerline{\bf Vectors make polynomial arithmetic $20\times$ faster!}

\end{frame}

\begin{frame}{Dense polynomials}

\begin{itemize}[<+->]

\item We arrived to {\tt Poly a} $\sim$ {\tt [Mono a]}
      by choosing a sorted list
      as a canonical representative of a class
      of equivalent polynomials.
\item Another choice of canonical representative
      is a polynomial of the same degree, but
      with monomials for all powers present.
\item It's enough to store coefficients alone:
      {\tt Poly a} $\sim$ {\tt [a]}.
\item This representation allows to implement
      asymptotically-superior algorithms:
      \begin{itemize}
      \item Karatsuba multiplication $O(n^{1.585})$,
      \item fast Fourier transform $O(n\log n)$.
      \end{itemize}
\end{itemize}

\pause

\begin{table}
\begin{tabular}{rrrr}
length & {\tt polynomial}, $\mu s$ & {\tt poly}, $\mu s$ & speedup
\\\hline
100 & 1733 & 33 & $52\times$
\\
1000 & 442000 & 1456 & $303\times$
\end{tabular}
\end{table}

\end{frame}

\begin{frame}[fragile]{User interface targeting REPL}

\begin{itemize}[<+->]
\item We need a nice way to input and output polynomials.
\item What about {\tt deriving (Show, Read)}?
\item Derived {\tt Show} looks nowhere close to
      a mathematical expression.

\begin{lstlisting}[language=Haskell]
Poly {unPoly = [(0,2),(1,-3),(2,1)]}
\end{lstlisting}

\item What about a hand-written {\tt Show}?
\begin{lstlisting}[language=Haskell]
> :set -XOverloadedLists
> [(2,1), (1,-3), (0,2)] :: VPoly Int
1 * X^2 + (-3) * X + 2
\end{lstlisting}
\item But writing correspondent {\tt Read} would be abysmal!
\item There is no point to support
\begin{lstlisting}[language=Haskell]
read "X^2 - 3 * X + 2"
\end{lstlisting}
if one can define {\tt X} as a pattern and write immediately
\begin{lstlisting}[language=Haskell]
X^2 - 3 * X + 2 :: Poly Int
\end{lstlisting}
\end{itemize}

\end{frame}

\begin{frame}[fragile]{Division with remainder}

\begin{itemize}[<+->]
\item We defined addition and multiplication, and
      subtraction poses no problem. What about division?
\begin{align*}
\quotRem~x^2~x &= (x, 0), \\
\quotRem~(x^2+1)~x &= (x, 1), \\
\quotRem~(x^2+1)~(x+1) &= (x-1, 2).
\end{align*}

\item Things get difficult for integral coefficients:
\begin{align*}
\quotRem~x~2.0 &= (0.5x, 0), \\
\quotRem~x~2\phantom{.0} &={} ?
\end{align*}

\end{itemize}

\pause

\begin{lstlisting}[language=Haskell]
quotRemPoly
  :: Fractional a
  => Poly a -> Poly a -> (Poly a, Poly a)
quotRemPoly = <long division algorithm>
\end{lstlisting}

\end{frame}

\begin{frame}[fragile]{Greatest common divisor}

\begin{itemize}[<+->]
\item Usually $\gcd$ is computed using Euclid's algorithm,
      which involves repeated $\quotRem$:
      $$ \gcd(70, 25) = \gcd(25, 20) = \gcd(20, 5) = 5. $$
\item Is it possible to implement $\gcd$ without access to
      $\quotRem$? E.~g., $\gcd(2x^2-2, 5x+5) = x+1$.
\item It appears that having access to $\gcd$ on coefficients
      is enough to implement $\gcd$ for polynomials.
\item Multiply, not divide!

\begin{align*}
\gcd(2x^2-2, 5x+5) & \sim \gcd(10x^2-10, 10x+10) \sim \\
& \sim \gcd(10x^2-10 - x(10x+10), 10x+10) \sim \\
& \sim \gcd(-10x - 10, 10x+10)
\sim 10x+10.
\end{align*}

\end{itemize}

\end{frame}

\begin{frame}[fragile]{What's wrong with {\tt Integral}?}

\begin{lstlisting}[language=Haskell]
class (Real a, Enum a) => Integral a where
  quotRem :: a -> a -> (a, a)
  toInteger :: a -> Integer
  ...
gcd :: Integral a => a -> a -> a
\end{lstlisting}

\pause

\begin{itemize}[<+->]
\item Restrictive superclasses,
      which has nothing to do with {\tt quotRem}.
\item Obnoxious {\tt toInteger}, which coupled
      with {\tt Num.fromInteger} means
      that only subrings of {\tt Integer}
      can be {\tt Integral}.
\item Function {\tt gcd} is constrained to domains,
      allowing division with reminder, which is
      too restrictive for polynomials.
\end{itemize}

\end{frame}

\begin{frame}[fragile]{{\tt Integral} done right}

\begin{lstlisting}[language=Haskell]
class Num a => GcdDomain a where
  gcd :: a -> a -> a
  default gcd :: (Eq a, Euclidean a) => a -> a -> a

class GcdDomain a => Euclidean a where
  quotRem :: a -> a -> (a, a)
\end{lstlisting}

\pause

\begin{lstlisting}[language=Haskell]
instance GcdDomain a => GcdDomain (Poly a) where
  ...
instance Fractional a => Euclidean (Poly a) where
  ...
\end{lstlisting}

\end{frame}

\begin{frame}[fragile]{Project structure}

\begin{columns}[onlytextwidth,T]
  \begin{column}{.4\linewidth}
  \vspace{4em}

  \begin{itemize}
  \item All actual code is in {\tt Internal} subtree.
  \item Public, user-facing modules
        are just dummies with re-exports.
  \item This approach helps a lot with circular dependencies
        between modules.
  \end{itemize}
  \end{column}
  \begin{column}{.1\linewidth}
  ~
  \end{column}
  \begin{column}{.5\linewidth}

\vspace{-1.3em}

\dirtree{%
.1 Data.
.2 Poly.
.3 Internal.
.4 Dense.
.5 DFT.hs.
.5 Field.hs.
.5 GcdDomain.hs.
.4 Dense.hs.
.4 Sparse.
.5 Field.hs.
.5 GcdDomain.hs.
.4 Sparse.hs.
.3 Semiring.hs.
.3 Sparse.
.4 Multi.hs.
.4 Semiring.hs.
.3 Sparse.hs.
.2 Poly.hs.
}

\end{column}
\end{columns}

\end{frame}

\begin{frame}{Package {\tt poly}}

\begin{itemize}[<+->]
\item Polynomials,
      polymorphic by coefficients and containers.
\item Full-featured:
      \begin{itemize}
      \item Dense and sparse representations.
      \item Laurent polynomials, allowing negative powers.
      \item Type-safe polynomials over many variables.
      \item Special polynomial sequences.
      \end{itemize}
\item GC- and cache-friendly implementation.
\item Unrivaled performance amongst Haskell native packages.
\item 1000+ tests, 98\% test coverage.
\end{itemize}

\bigskip
\bigskip

\pause

\centerline{\Huge\bf Thank you!}

\bigskip

\begin{table}
\begin{tabular}{rl}
\faAt\ 1@dxdy.ru & \faTelegram\ Bodigrim \\
\faGithub\ github.com/Bodigrim/poly & \faGithub\ github.com/Bodigrim/my-talks
\end{tabular}
\end{table}

\end{frame}

\end{document}
